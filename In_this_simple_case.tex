In this simple case, 
you might not feel the convenience of the user-defined function, 
but you will start to feel its power as you start writing longer codes. 
Advantages of using \ilcom{function} are

\begin{enumerate}
\item Once written in a macro file, it could be used as a single line function 
as many times as you want in the macro file. This also means that if there is a bug, 
fixing the function solves the problem in all places where the function is used.
\item Long codes could be simplified to an explicit outline of events. Such as:
\begin{lstlisting}[numbers=none]
macro "whatever" {
    	function1;
		function2;
		function3;
}
\end{lstlisting}
\end{enumerate}

\subsection{String Arrays}
Array is a powerful tool. before going into how to use it, here is an easy explanation. 
Imagine that an array is a stack of boxes. Boxes could contain either numbers or strings. 
For instance, if you have a following list of strings: