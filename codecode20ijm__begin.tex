(code/code20.ijm)

\begin{itemize}
\item Line 3 uses a function that creates a new array (\ilcom{newArray()}),
defined by a parameter for number of array elements (in the example case its 5) and its name \ilcom{EMBL}.
\item From line 4 to 8, each array from position 0 to 4 will be filled with
names (Array starts with 0th element).
\item Line 9 asks the user to input the address (position) within the array.
Then this input address is examined if the address exists within the
\ilcom{EMBL} array in line 10. \ilcom{EMBL.length} returns the number of "boxes"
within the array. If this is satisfied, then line 10 prints out the string in that address.
\end{itemize}

Array could be created and initialized with actual values at the
same time, so line 3 to 8 could be written in a single line like this: