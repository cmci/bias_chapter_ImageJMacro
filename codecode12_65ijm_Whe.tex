(code/code12_65.ijm)
When you run this code as it is, line 4 and line 8 are both executed and prints
the messages. For the first \ilcom{if} parenthesis, \ilcom{\&\&} operator tests if
both sides are true. If both are indeed true, it returns true (1), and that is
the case above. If one of them or both are false, then \ilcom{\&\&}
operator returns false(0).

On the other hand, in the second if parenthesis,
\ilcom{||} operator tests if one of the two sides is true. Since both are
true in the above code, OR operator returns true because at least one of them is
true. Only when both sides are false, the returned value becomes false (0).

\begin{indentexercise}
{1}
Adjust the values of \ilcom{a} and \ilcom{b} in code 12\_65 to \ilcom{true} or \ilcom{false} and
compose other three possible combinations (e.g. \ilcom{a = true}, \ilcom{b = false} will print
only one line). Check the output.

Next, change the values of \ilcom{a} and \ilcom{b} to 0 and/or
1 and check the results. 
\end{indentexercise}

Here is a more realistic example (though not very useful), an extended version
of code 16\_6.