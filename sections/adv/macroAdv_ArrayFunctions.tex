\subsection{Array Functions}

Arrays could be directly treated using array functions. Since array is a very usable form of holding numbers and strings, it's good for you to know what they could do. Here is the list. 

\begin{shaded}\begin{indentCom}
\item \textbf{Array.concat(array1,array2)} Returns a new array created by
joining two or more arrays or values. 
\item \textbf{Array.copy(array)} Returns a copy of array. 
\item \textbf{Array.fill(array, value)} Assigns the specified numeric value to
each element of array.
\item \textbf{Array.findMaxima(array, tolerance)} Returns an array holding the peak positions (sorted with descending strength). Tolerance is the minimum amplitude difference to needed to separate two peaks. There is an optional 'excludeOnEdges' argument that defaults to 'true'. Examples. Requires 1.48c.
\item \textbf{Array.findMinima(array, tolerance)} Returns an array holding the minima positions. Requires 1.48c.
\item \textbf{Array.fourier(array, windowType)} Calculates and returns the Fourier amplitudes of array. WindowType can be "none", "Hamming", "Hann", or "flat-top", or may be omitted (meaning "none"). See the TestArrayFourier macro for an example and more documentation. Requires 1.49i. 
\item \textbf{Array.getStatistics(array, min, max, mean, stdDev)} Returns the
min, max, mean, and stdDev of array, which must contain all numbers.
\item \textbf{Array.print(array)} Prints the array on a single line. 
\item \textbf{Array.rankPositions(array)} Returns, as an array, the rank
positions of array, which must contain all numbers or all strings. 
\item \textbf{Array.resample(array,len)} Returns an array which is linearly resampled to a different length. Requires 1.47j. 
\item \textbf{Array.reverse(array)} Reverses (inverts) the order of the
elements in array. 
\item \textbf{Array.show(array)} Displays the contents of array in a window. Requires 1.48d.
\item \textbf{Array.show("title", array1, array2, ...)} Displays one or more arrays in a Results window (examples). If title (optional) is "Results", the window will be the active Results window, otherwise, it will be a dormant Results window (see also IJ.renameResults). If title ends with "(indexes)", a 0-based Index column is shown. If title ends with "(row numbers)", the row number column is shown. Requires 1.48d. 
\item \textbf{Array.slice(array,start,end)} Extracts a part of an array and
returns it. 
\item \textbf{Array.sort(array)} Sorts array, which must contain all numbers
or all strings. String sorts are case-insensitive in v1.44i or later.
\item \textbf{Array.trim(array, n)} Returns an array that contains the first n
elements of array.
\end{indentCom}\end{shaded}

For example, array could be sorted and reversed. Try the following codes. 

\begin{lstlisting}[numbers=none]
EMBL = newArray("Heidelberg","Hamburg","Hixton","Grenoble","Monterotondo");
Array.print(EMBL);
Array.sort(EMBL);
Array.print(EMBL);
Array.reverse(EMBL);
Array.print(EMBL);
\end{lstlisting} 
The output of this code is:
\begin{lstlisting}[numbers=left]
Heidelberg,Hamburg,Hixton,Grenoble,Monterotondo
Grenoble,Hamburg,Heidelberg,Hixton,Monterotondo
Monterotondo,Hixton,Heidelberg,Hamburg,Grenoble
\end{lstlisting} 
The first line is printed in the order when the array was initialized. After
sorting, names are in alphabetical order. Third line shows the reversed
elements. 

Some functions return an array rather than taking array/s as argument. See Appendix for a list of those functions (\ref{subsec:arrayreturn}).
