(code/code20_5.ijm)

\begin{itemize}
\item Line 3: Check if the selection type is a straight line ROI. If not, macro terminates leaving a message. 

\begin{indentCom}
\textbf{selectionType}()\\ 
Returns the selection type, where 0=rectangle, 1=oval, 2=polygon, 3=freehand, 4=traced, 5=straight line, 6=segmented line, 7=freehand line, 8=angle, 9=composite and 10=point. Returns -1 if there is no selection.
\end{indentCom}

\item Line 4: Empty array \ilcom{tempProfile} is loaded with the intensity profile along the line ROI by \ilcom{getProfile}().
\item
\begin{indentCom}
\textbf{getProfile}()\\
Runs \ijmenu{[Analyze > Plot Profile]} (without displaying the plot) and returns the intensity values as an array.
\end{indentCom}

\item Line 5: Passing the array \ilcom{tempProfile} to function "output\_results", which prints the content of array in the table shown in the ``Results'' window. 

\item Line 7 to 14: A function for outputting the profile array in the table shown in the ``Results'' window. It takes an argument \ilcom{rA}, which is supposed to be an array. 
\item Line 8: Clears the results table. 
\item Line 9 to 12: for-loop to go through the array and to print out each element. 
\item Line 10: Sets the pixel position along the segment in the column labeled "n". 
\item Line 11: Sets the content of the array (pixel intensity) in the column labeled "intensity".

\begin{indentCom}
\textbf{setResult}("Column", row, value)
Adds an entry to the ImageJ results table or modifies an existing entry. The first argument specifies a column in the table. If the specified column does not exist, it is added. The second argument specifies the row, where 0<=row<=nResults. (nResults is a predefined variable.) A row is added to the table if row=nResults. The third argument is the value to be added or modified. 
\end{indentCom}
\item Line 13: Updates the table shown in the ``Results'' window. 

\begin{indentCom}
\textbf{updateResults}()
Call this function to update the "Results" window after the results table has been modified by calls to the setResult() function. 
\end{indentCom}
\end{itemize}

\begin{indentexercise}
{1}
Modify code 20.5 that the macro calculates the sum of all intensities.\\

Hint:

\begin{itemize}
\item
\item You do not need the function anymore. 
\item \ilcom{for}-loop should be used.
\item Use \ilcom{tempProfile.length}
\end{itemize}

\item \textbf{Answer}: The key for getting the sum of values in an array is for-loop to go through all elements of the array. The total sum of array values is calculated by adding up values during this for-loop.   
\begin{lstlisting}[numbers=none]
macro "get profile and printout" {
	if (selectionType() !=5) exit("selection type must be a straight line ROI");
	tempProfile=getProfile();
	sum = 0;
	for (i = 0; i < tempProfile.length; i++){
		sum += tempProfile[i];
	}
	print("sum of values:", sum);
}  
\end{lstlisting}

Another way of achieving the similar task is by using Array related function. We will see this later. 
\end{indentexercise}

\subsection{Array Functions}

Arrays could be directly treated using array functions. Since array is a very usable form of holding numbers and strings, it's good for you to know what they could do. Here is the list.

\begin{shaded}
\begin{indentCom}
\item \textbf{Array.concat(array1,array2)} Returns a new array created by
joining two or more arrays or values. 
\item \textbf{Array.copy(array)} Returns a copy of array. 
\item \textbf{Array.fill(array, value)} Assigns the specified numeric value to
each element of array.
\item \textbf{Array.findMaxima(array, tolerance)} Returns an array holding the peak positions (sorted with descending strength). Tolerance is the minimum amplitude difference to needed to separate two peaks. There is an optional 'excludeOnEdges' argument that defaults to 'true'. Examples. Requires 1.48c.
\item \textbf{Array.findMinima(array, tolerance)} Returns an array holding the minima positions. Requires 1.48c.
\item \textbf{Array.fourier(array, windowType)} Calculates and returns the Fourier amplitudes of array. WindowType can be "none", "Hamming", "Hann", or "flat-top", or may be omitted (meaning "none"). See the TestArrayFourier macro for an example and more documentation. Requires 1.49i. 
\item \textbf{Array.getStatistics(array, min, max, mean, stdDev)} Returns the
min, max, mean, and stdDev of array, which must contain all numbers.
\item \textbf{Array.print(array)} Prints the array on a single line. 
\item \textbf{Array.rankPositions(array)} Returns, as an array, the rank
positions of array, which must contain all numbers or all strings. 
\item \textbf{Array.resample(array,len)} Returns an array which is linearly resampled to a different length. Requires 1.47j. 
\item \textbf{Array.reverse(array)} Reverses (inverts) the order of the
elements in array. 
\item \textbf{Array.show(array)} Displays the contents of array in a window. Requires 1.48d.
\item \textbf{Array.show("title", array1, array2, ...)} Displays one or more arrays in a Results window (examples). If title (optional) is "Results", the window will be the active Results window, otherwise, it will be a dormant Results window (see also IJ.renameResults). If title ends with "(indexes)", a 0-based Index column is shown. If title ends with "(row numbers)", the row number column is shown. Requires 1.48d. 
\item \textbf{Array.slice(array,start,end)} Extracts a part of an array and
returns it. 
\item \textbf{Array.sort(array)} Sorts array, which must contain all numbers
or all strings. String sorts are case-insensitive in v1.44i or later.
\item \textbf{Array.trim(array, n)} Returns an array that contains the first n
elements of array.
\end{indentCom}\end{shaded}

For example, array could be sorted and reversed. Try the following codes.