(code/code12_2.ijm)
The output is now 0, indicating that ``\ilcom{(5 == 4)}'' is
0.
What double equal signs \ilcom{==} are doing in these
examples are comparison of numbers in the left and the right side, and if
the numbers are the same, it returns 1 and if they are not the same, it returns 0. 1 and
0 actually are representing \textbf{true} (= 1) or \textbf{false} (= 0), the
\textbf{boolean values} .

We could also test if they are NOT equal. For this, replace \ilcom{==} by
\ilcom{!=}.