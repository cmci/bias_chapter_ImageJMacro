\begin{itemize}
\item line 3: The macro interpreter first assigns 0 to the counter.
\item line 4: The interpreter evaluates if the counter value is less than or equal to 90. Since counter is initially 0\ldots 
\item line 5: Printing function is executed. 
\item line 6: counter is added with 10. 
\item line 7: the interpreter realizes the end of "while" boundary and goes back to line 4. Since counter= 10 <= 90, line 5 is again executed\ldots and so on. When counter becomes 100 in line 6 after several more loops, counter is no longer <=90. So the interpreter goes out from the loop, moves to line 8. Then the macro is terminated.
\end{itemize}

Line 6 could be written in the following way as well.