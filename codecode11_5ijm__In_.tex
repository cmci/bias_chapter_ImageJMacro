(code/code11_5.ijm)

In this example, the exit condition for going out from looping is \ilcom{counter < 0}, and the initial value of \ilcom{counter} is 0, which does not satisfy that looping condition. Since this evaluation occurs only after the looping part is executed for the first time, the macro still prints out a line before it exits from the loop.

Condition for the while-statement could be various. Here is a small list of comparison operators.

\begin{indentCom}
 \begin{tabular*}{0.5\textwidth}{ l r }
< & less than \\
<= & less than or equal\\ 
> & greater than\\ 
>= & greater than or equal to\\
== & equal\\
!= & not equal\\
 \end{tabular*}
\end{indentCom}

\begin{indentexercise}
{3}
Modify code 11 so that the macro prints out numbers from 200 to 100, with an increment of -10. 

\item \textbf{Answer}: There could be slightly various ways to do this modification, but here is one way. 
	\begin{lstlisting}
macro "while looping1" {
	counter=200;
	while (counter>=100) {
		print(counter);
		counter -= 10;
	}
}
\end{lstlisting}

\end{indentexercise}

\subsubsection{Why is there while-loop?}

An often raised question with the while-loop is why do we have two types of loops, 
the for-loop and the while-loop. Answering to this question, they have different
flexibility. The for-loop is rather solid and the while-loop is more flexible. In the
example code below, the user is asked for a correct number and if the answer is wrong, the
question is asked 5 times repeatedly. Number of loop is not determined by the
programmer, but interactively when the code is running. We will study
the branching of the program based on if-else in the next section.